\documentclass[10pt]{article}
\usepackage[T1]{fontenc}
\usepackage{titling}
\usepackage{booktabs}
\usepackage{amsmath} 
\setlength{\droptitle}{-12em}   % This is your set screw
\title{EERI418 Pracitcal experiment design}
\author{MJ Bezduienout\\24162299 \and G.A. de Klerk \\23425733}
\begin{document}
\maketitle


\begin{table}[h]
\centering
\caption{Values to be measured/calculated}
\label{my-label}
\begin{tabular}{@{}ll@{}}
\toprule
Variables and parameters & Symbol and unit                          \\ \midrule
Armature Volatage        & $v_a(t)${[}V{]}                          \\
Armature Current         & $i_a(t)${[}A{]}                          \\
Motor Speed              & $\omega(t) = \dot{\theta}(t)${[}rad/s{]} \\
Armature Resistance      & $R_a$ {[}$\Omega${]}                     \\ \bottomrule
\end{tabular}
\end{table}

The purpose of this lab session is to design experiments to determine the time constants $\tau_a$and $\tau_l$, where:\\
\begin{align*}
\tau_a & = \frac{L_a}{R_a} \\
\tau_l & = \frac{R_aJ}{R_ab+K_bK_m} 
\end{align*}
This will be done using two experiments:
\begin{itemize}
	\item First, the motor's speed will be measured as it slows down. This makes it possible to determine the moment of innertia and the mechanical time constant
	\item Second, the Speed will be realted to different values for the armature current and voltage.
\end{itemize}
These values will be digitally captured and used to determine the nessecary parameters. 

\end{document}
